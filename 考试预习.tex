\documentclass[a4paper, UTF8]{ctexart}
\usepackage{geometry}
\usepackage{fancyhdr}
\usepackage{setspace}
\usepackage{titlesec}

\geometry{top=2.6cm, bottom=2.6cm, left=2.45cm, right=2.45cm, headsep=0.4cm, foot=1.12cm}
\onehalfspacing

\pagestyle{fancy}
\lhead{退学启程考试预习}
\chead{}
\rhead{2019-2020学年秋学期}
\lfoot{}
\cfoot{\thepage}
\rfoot{}

\title{\huge{\heiti 退学启程考试预习}}
\author{文先森}
\date{}

\setcounter{section}{-1}
\titleformat{\section}[block]{\LARGE\bfseries}{第\arabic{section}章}{1em}{}[]

\setlength{\headheight}{15pt}

\begin{document}
\maketitle

\section{前言和绪论}
		(2018年第2版)中国的创业活动在全球中处于相对活跃状态,从以生存型创业为主转变为以\textbf{机会型创业}为主。

		(2005年第1版)中国在全球的创业活动中处于活跃状态,但是以生存型为主;中国的创业机会多,国民的创业动机比较强,但创业能力不足。

		创业精神不仅在创办新企业时需要,大企业、非营利机构同样需要创业精神。

		国外大学的管理教育朝着创业型管理教育演变,创业教育的范围不局限于商学院中的管理教育,一些理工科院系也积极探索和开展创业教育。

		(绪论)创业者常问的问题,背后本质是创业的规划与战略问题。

		创业理想背后是所谓的\textbf{创业动机}。创业理想分为多个类型,有志存高远,以企业上市为目标,以推出改变人们日常生活和工作方式、
		改变世界的产品和服务作为努力的方向,实现自身财务自由的同时,还能功成名就,影响世界;
		不甘一生平庸,将创业作为实现自己想法和价值的途径,追求个人成就感的满足;希望通过创业给自己争取更多的自由空间和选择权。

		\textbf{创业决策分析}的蝴蝶模型,包括四个方面:创业者的个性和动机,创业者的知识、能力、经验、资源和网络,
		创业机会的潜在价值,创业者的学习意愿和能力。

		\textbf{创业过程的三要素}:机会、资源和创业团队。创业团队的执行力,一方面取决于创业团队的组成,另一方面取决于创业企业的制度安排。
		制度安排分为三个层次:治理层面,股权和控制权的配置;日常管理制度层面;非正式的制度安排以及文化氛围。

		\textbf{创业失败}的外部因素:制度环境、宏观经济环境、技术创新的周期、行业的成长性和特定区域的创业环境政策、运气。
		内部因素:企业的价值创造力和制度安排。

\section{创业与创业精神}
	\subsection{创业的概念和类型}
		创业的本质是有价值的机会与具有创业精神的人之间的结合,创业是一个过程。

		创业的分类:市场与产品/服务两个维度,每个维度分为成熟和新兴两类。

	\subsection{创业、创业精神与经济增长}
		19世纪中后期的美国,\textbf{规模}成为主旋律,大规模生产以大规模销售为前提,大规模销售以大规模消费为前提。
		在这种增长模式下,企业围绕扩大规模组织实施企业行为,一方面对生产技术进行投资或通过横向并购获得技术,以获得成本优势;
		另一方面充分利用闲置产能,通过多元化方式进入相关市场。20世纪的美国,技术革命催生了一批有创新和创业精神的企业。
		企业并购不再是围绕扩大生产规模和市场范围,而是为获取技术与专利、领先的用户、创新的思想与创业的团队,保持创新能力和创业精神。
		\textbf{创新和创业}取代了规模成为经济增长的主旋律。
	
		\textbf{机会型创业相对于生存型创业能带来更多的就业机会、新市场机会、创新机会和企业增长机会}。

		许多创业活动直接诞生的是对国家经济命脉有着重要影响的世界级企业。

		成熟企业内部的创业活动也日益成为经济增长的推动器。

	\subsection{创业精神与社会发展}
		在专业化分工和组织规模极大地提升了社会产出效率的过程中,人类自身感受到的幸福感并没有得到同样的提高。
		在企业获得利润的同时,能否同时帮助人类获得心灵的解放和个性的张扬,是工业社会永远无法回答的问题。

		创新精神和创业活动的兴起将“人”推向了整个社会发展的中心。创业活动使企业工作的重点从对人的管理,转移到对人类潜质的开发;
		从强调员工的服从,转移到鼓励员工创新;从强调企业文化对人的影响,转移到帮助人类获得心智的模式转变。
		创业和创业活动首先承认人的天赋和能力,社会和企业需要做的就是发掘与充分利用人的这种能力和天赋。

		创业活动使人们能够逐渐从工作本身获得满足感和成就感。创业活动往往能够将工作过程与工作成果之间的联系体现得更加清晰。
		多样化的工作给创业活动的参与者以更大的幸福感。学习成为满足人们好奇心和解决问题的手段。

		创业活动充分保障了社会良好的流动性。

	\subsection{创业环境}
		GEM研究将影响创业活动的环境条件分为一般环境条件和创业环境条件(也称为“创业生态系统”)。
		一般环境条件包括:国家的对外开放程度、政府职能、企业管理水平和技能、技术研发水平和程度、基础设施、资本市场、劳动力市场、制度完善程度等。
		创业生态系统包括11个方面:创业融资、政府政策、税收和行政体制、政府项目支持、在校创业教育和培训、离校创业教育和培训、研究与开发效率、商业与法律基础设施、国内市场动态性、有形服务的基础设施、文化和社会规范。

		有三个宏观环境因素将会在未来十年中高度影响中国创业活动的数量和质量。
		一是整体经济的增长态势及增长方式转变;二是全球化浪潮与互联网经济;三是部分政策性垄断产业的进一步放开。
	
	\subsection{创业教育}
		人力资本是推动新时代经济社会发展的关键要素。创新创业教育的本质意义在于激发人的主动性和创造性,提升创业者的认知水平和创新创业技能,培育企业家精神和团队精神;
		创新创业教育,最重要的是培养“人”,而不能简单片面地强调培养项目和孵化企业。

	\subsection{课后习题}
		\textbf{对比过去20年间美国、日本和中国的创业活动,分析其对国家经济和社会发展的作用。}



		\textbf{观察你身边的3位创业者,列举出他们所显现的可i通过教育或者其它方式传递的特质,同时也列举出你认为无法通过教育方式传递的特质。}



		\textbf{你认为创业活动会通过哪些方式影响经济增长和社会发展?}


\section{创业过程}
	\subsection{创业的一般过程}
		创业过程指的是创业者发现和评估商机,并且将商机转化为创业者对新创企业进行成长管理的过程。创业管理和一般的企业职能管理相比,涵盖的时间更为漫长,涉及对因素更为复杂。

		完整的创业过程,通常按时间顺序划分为三个阶段:机会识别、创办新企业和新创企业的成长管理。

		创业机会识别:创意挖掘(不确定性,市场前景未知),机会识别(具备市场价值),商业模式设计。

		创办新企业:组建创业团队,开发商业计划(产品技术,盈利模式,市场前景,团队组成),创业融资(内源式,外源式债务融资和股权融资)。
		\emph{第一起国内上市公司参与风险投资:上海一百投资视美乐。}

		成长管理:战略管理(抓住自己和市场上已有企业的差异,形成自己独特的竞争优势),成长管理(领导人是关键),危机管理。

	\subsection{新创企业的生命周期}
		创业的生命周期分为种子期、初创期、发展期和成熟期。

		新企业的建立,标志着创业者成功度过种子期。初创期资金枯竭,只能出售企业或者破产。
		发展期面对层出不穷的问题,创业者需要建立一套合理的管理制度。
		成熟期技术风险和市场风险降低,管理风险增大,如何保持企业竞争力、这进行多元化经营管理是主要问题。

	\subsection{创业中所需的资源}
		创业资源分为两类:要素资源(场地、资金、人才、管理、科技)和环境资源(政策、信息、文化、品牌)。

	\subsection{精益创业的理念和思维}
		精益创业的核心思想是先在市场中投入一个最简化可行产品,然后通过不断的学习和有价值的用户反馈,对产品进行快速迭代优化,以期适应市场。

		精益创业可以分为探索(客户发现->客户确认/转型)和执行(客户创造->公司建立)两个阶段。

	\subsection{三个重要的创业模型}
		Timmons创业过程模型的核心思想在于,创业是一个高度的动态过程,其中商机、资源、创业团队是创业过程最重要的驱动因素。

		Sahlman创业模型认为要把握四个关键要素:人、机会、外部环境、创业者的交易行为。

		创业行动模型提出了创业企业获取正当性的三种战略:杠杆、协调、制定。

	\subsection{课后习题}
		\textbf{创业过程通常由几个阶段组成,每个阶段创业者需要注意哪些主要问题?}

		\textbf{创业资源与一般企业所需要的资源有什么联系和区别?列举你觉得最重要的三项创业资源并予以解释。}

		\textbf{调研几个创业案例,用Timmons模型或者Salhman模型来解释这些新创企业的发展过程。}

		\textbf{对创业企业进行访谈,调研其创业过程中的主要创业行动,了解正当性对创业企业获取资源的影响,并调研创业企业是如何获取和管理正当性的。}

\section{创业机会}
	\subsection{创业机会的概念和重要性}
		Timmons认为,创业过程的核心是创业机会问题,创业过程是由机会驱动的。

		德鲁克指出,在产品市场的创业活动有三大类机会:
		由于新技术的产生,创造新信息;
		由于时间和空间的原因导致信息不对称而引起市场失效,利用市场失灵;
		当政治、管制和人口统计特征发生重要变化,与资源利用相关的成本和利益便会发生转变,这种转变可能创造机会。

		创业机会是指在新的生产方式、新的产出或新的生产方式与产出之间的关系形成过程中,
		引进新的产品、服务、原材料和组织方式,
		得到比生产成本更高价值的情形。

		创业机会->商业概念->商业模式

		\textbf{创业机会存在的原因}
		
		经验解释:法规的变化,巨大的行业变化,价值链和分销渠道的重构,知识产权的优势,现有的人员、资本和管理的使用不当,创业精神,市场领导者不能满足或漠视顾客的需要。

		福利经济学解释:不能实现帕累托最优就是市场的失灵和失败,存在创业机会以实现潜在的帕累托改进。
		发现并利用信息、打破垄断、提高公共产品的私有化程度、创造外部性的市场。

		基于非均衡理论的解释:人们对资源价值的判断不同是创业机会出现的必要条件,
		现有的价格不能完全反映与资源有关的信息,
		未来的信息无法完全还原到现有的价格信息上,
		创业决策并不都是最优决策,
		价格不能实时反映资源的生产力,不能自动实现变更。
		创业机会的出现,要求创业者与资源所有者和其他创业者的价值判断不同。

		基于社会学理论的解释:创业者和制度环境的相互作用。制度创业、制度创业者。

	\subsection{创业机会的期望价值:选择利于创业的机会}
		机会出现的产业:知识因素(研究与开发密集、技术创新主要来源公共部门、较小规模的单位可实现创新)、需求因素(市场规模、市场成长性、细分情况)、
		产业生命周期(通行标准出现前)、产业结构(资本密集程度低、规模经济效应不显著、产业集中程度低、中小企业为主)。

		机会出现的时间:信息的扩散和利润诱惑的减少,将降低人们追求某具体机会的动力。
		限制其他创业者模仿的机制、减缓信息扩散的速度、其他人无法模仿、替代、交易或获得稀有的资源。
		预测和判断“机会之窗”:寻找产业发展“瓶颈”的价值链环节。

		机会的类型:市场需求是否已知/资源和能力是否确定

	\subsection{创业机会评价}
		阶段性决策方法:在机会开发的每个阶段都进行机会评价。

		影响机会评价标准的四个因素:创业经历、行业经验、管理经验、行业的新兴程度。

		Timmons的机会评价框架:行业与市场、经济因素、收获条件、竞争优势、管理团队、致命缺陷、创业家的个人标准、理想与现实的战略性差异。

	\subsection{创业机会的识别与开发}
		创业机会识别与开发的三个过程:感知、发现、创造。

		创业者的重要资源:人力资本、机会识别能力、社会资本

	\subsection{课后习题}
		\textbf{讨论创业机会的静态概念和动态概念的区别和联系。}

		\textbf{分析互联网创业浪潮中取得成功的一些创业者,探讨:他们曾经面临哪些主要的机会?帮助他们取得创业成功的核心因素有哪些?他们之间共性的特点有哪些?}

		\textbf{在当前“大众创新,万众创业”新形势下,国家出台了一系列支持创新和创业的政策,请结合你感兴趣的领域或行业,分析其中会出现什么样的创业机会?你认为创业者需要在哪些方面做准备,才能抓住这样的创业机会?}

\section{商业模式}
	\subsection{商业模式的概念}
		商业模式的核心三要素是顾客、价值和利润。

	\subsection{商业模式的作用}
		促使创业者缜密地思考市场需求、生产、分销、企业能力、成本结构等各方面的问题,将商业的所有元素协调成一个有效、契合的整体;
		让顾客清晰了解企业可能提供的产品和服务,实现企业在顾客心目中的目标定位;
		让企业员工全面理解企业的目标和价值所在;
		让股东更清晰、方便地判断企业的价值及其在市场中的地位变化。

		商业模式的逻辑主线:价值的创造、传递、分享

		企业的核心价值:产品、价格、渠道、服务和体验

		商业模式考察由建立和运营企业所必须的各个环节紧密构成的完整要素链。
		创业战略则在于规划一条从创业机会、组织资源通向创业目标的道路,必须考虑竞争因素。

	\subsection{商业模式的评价}
		商业模式四个核心内容:收入来源、成本的主要构成、所需要的投资额、关键的成功要素

		商业模式合理性的基本检验方法:
		逻辑检验、数字检验

		商业模式评价的3S原则:简明、可扩展、可持续

		国内成功的商业模式的常见类型:复制、替代、革命性创新

	\subsection{商业模式的演进与持续创新}
		从模糊到清晰,从“复制”到“本土化”

		与专注于降低成本的方式相比,通过改善商业模式对提高企业在行业中的地位效果更为显著。

		创新五个方面:开发新的产品,开发新的生产程序,采用新的组织方式,开辟新的市场,采用新的原材料

	\subsection{课后习题}
		\textbf{对中国开展共享单车业务的不同创业企业的商业模式和竞争战略进行比较分析,预测前景。}

		\textbf{与不同创业阶段的创业者进行交流,了解其商业模式是如何形成和调整的。}

\section{创业者与创业团队}
	\subsection{创业者心理与性格特质}
		胸有抱负,目标明确;富有创新,自我激励;自信乐观、百折不挠;团队精神、善于学习;诚实正直,精力充沛

	\subsection{创业者的个人能力}
		形成创业文化的技能:人际沟通和团队工作技能、领导技能、帮助督导和矛盾管理技能、团队工作和人员管理技能
		
		管理或技术才能:两大跨职能领域(行政管理和法律税收)、四大关键职能领域(市场营销、金融、生产运营、财务)

	\subsection{创业团队}
		狭义的创业团队是指有着共同目的、共享创业收益、共担创业风险的一群经营新成立的营利性组织的人,即企业的共同创始人。

		广义的创业团队是指共同创始人和早期重要员工。

		创业团队的组建方式:相似性、互补性、渐进性、动态性

	\subsection{股权分配}
		股权分配的核心问题是剩余索取权和控制权的配置。

		创业企业的报酬制度包括股票、薪金和补贴等经济报酬以及其他一些非经济报酬,如实现个人发展和个人目标、培养技能等。

		合理分配报酬:形成分享财富的理念、综合考虑企业与个人目标、规范制定报酬制度的程序、实施合理分配方案、综合考虑分配时机和手段、适时采用股票托管协议

		股票期权计划与准股票期权计划

		团队成员绩效评价:创业思路、商业计划准备、敬业精神和风险、工作技能经验业绩记录或社会关系、岗位职责

	\subsection{课后习题}
		\textbf{如何理解创业者?}

		\textbf{举例说明创业者的个性特征和能力对企业的影响。}

		\textbf{在创建创业团队的时候应该注意哪些方面的问题?如何解决团队成员的稳定性和流动性矛盾?}

		\textbf{当团队成员对企业目标的理解出现分歧时,如何保障团队的凝聚力?}

		\textbf{如何结合创业团队的实际设计合理的激励机制和报酬体系?}

\section{商业计划}
	\subsection{商业计划}
		从三个方面考虑问题:市场、投资者和创业者自身

	\subsection{商业计划书}
		为什么要有商业计划:创业融资、认识自己、战略思考、创建和凝聚团队、取得政府和相关机构的支持

		吸引风险投资商的商业计划:产业和市场、产品和技术、风险和盈利、管理和组织、竞争战略,资金需求、使用、回报、退出措施

		吸引合伙人:出资方式、利益分享机制

		政府:经济和社会意义

	\subsection{商业计划书的主要内容}
		写作步骤:将商业计划构想细化、市场调研、商业计划书写作、商业计划书的检查和调整、商业计划书的推介

	\subsection{商业计划书的常见问题}
		执行总结太长、过分强调技术、缺乏市场分析和竞争对手分析、过于乐观对风险及相应对策考虑不足

		六大要素:商业模式、市场、产品、竞争、管理团队、行动

		基本要求:力求准确、简明扼要、条理清晰、注意语言、强调可信性

		4P:产品战略、定价战略、渠道战略、促销战略
	\subsection{课后习题}
		\textbf{商业计划的构成框架及其关键要素包括哪些?}

		\textbf{对于不通发展阶段、不同行业的创业企业,商业计划在写作上应该有哪些侧重?}

\end{document}